\addtocontents{toc}{\protect\thispagestyle{empty}} % força o "empty" na primeira pagina do sumário
\pagestyle{empty} % remove a nunemracao das demais paginas de sumario (se houver mais de uma) 

%apresenta ate o nivel 5, de paragrafo
\setcounter{tocdepth}{4}

%apresenta a numeracao ate o nivel 5, de paragrafo
\setcounter{secnumdepth}{4}

% titletoc, configura a formatação do sumário
\titlecontents{chapter}[10pt]
  {\normalfont\normalsize\bfseries}
  {\contentslabel{10pt}\uppercase} % removido \MakeUppercase
  {}
  {\titlerule*[0.4pc]{.}\contentspage} % aqui sim!

\titlecontents{section}[17pt]
  {\normalfont\normalsize}
  {\contentslabel{17pt}\uppercase}
  {}
  {\titlerule*[0.4pc]{.}\contentspage}

\titlecontents{subsection}[30pt]
  {\normalfont\normalsize\bfseries}
  {\contentslabel{30pt}}
  {}
  {\titlerule*[0.4pc]{.}\contentspage}

\titlecontents{subsubsection}[45pt]
  {\normalfont\normalsize\bfseries\itshape}
  {\contentslabel{45pt}}
  {}
  {\titlerule*[0.4pc]{.}\contentspage}

\titlecontents{paragraph}[53pt]
  {\normalfont\normalsize\itshape}
  {\contentslabel{53pt}}
  {}
  {\titlerule*[0.4pc]{.}\contentspage}


%Altera e Formata o titulo do sumario
\renewcommand{\contentsname}{\normalsize\centerline{\bfseries\uppercase{Sumário}}}

\tableofcontents % Insere o sumário
\clearpage % Garante que o sumário termine aqui
\pagestyle{fancy} % Retorna ao estilo "fancy" para o restante do documents