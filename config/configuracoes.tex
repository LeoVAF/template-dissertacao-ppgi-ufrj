\usepackage[brazil]{babel}              % Traduções utilizados por vários pacotes
\usepackage[top=3cm,                    % definição padrao das margens das paginas
            bottom=2cm,
            left=3cm,
            right=2cm]{geometry}
\usepackage[T1]{fontenc}		        % Selecao de codigos de fonte.
\usepackage[utf8]{inputenc}		        % Codificacao do documento UTF-8 
\usepackage{PTSerif}                    % Pacote da fonte escolhida: Opções de fontes em: https://ctan.org/topic/font
\usepackage{graphicx}		            % Inclusão de gráficos, pdfs
\usepackage{lscape}                     % permite colocar um conteudo em formato paisagem (rotacionar)
\usepackage{color}                      % Permite utilizar cores pelos nomes
\usepackage{microtype}     			    % para melhorias de justificação
\usepackage[indentfirst=false]{quoting} % Usado para citação
\usepackage{url}                        % Usado para formatar URLs no texto
\usepackage{enumitem}                   % Permite alterar o formato das listas enumeradas 
\usepackage{booktabs}                   % Usado para personalizar elementos em tabelas, como \toprule e \midrule
\usepackage{arydshln}                   % Usado para personalizar elementos em tabelas, como \cdashline
\usepackage{amsmath}                    % Usado nas referencias de equações, através do comando \eqref
\usepackage{tocloft}                    % Pacote para formatar sumário, lista de figuras e tabelas
\usepackage{float}                      % Necessario para posicionar as figuras e tabelas de forma flutuante
\usepackage{enumitem}                   % Utilizada para formatar uma lista enumerada 
\usepackage{svg}                        % Pacote que permite utilização de imagens SVGs (vetoriais) no documento 
\usepackage{adjustbox}



%%%%% Redefinindo localização da paginação - INICIO %%%%%%%%%%%%%%%%%%%%%%%%
\usepackage{fancyhdr} % Pacote utilizado para alterar a posicao da paginação
\pagestyle{fancy}
\fancyhf{} % Limpa todos os cabeçalhos e rodapés padrão
\fancyhead[R]{\thepage} % Coloca o número da página no canto superior direito
\renewcommand{\headrulewidth}{0pt} % Remove a linha do cabeçalho (opcional)
\fancypagestyle{plain}{
    \fancyhf{} % Limpa cabeçalhos e rodapés
    \fancyhead[R]{\thepage} % Número da página no canto superior direito
    \renewcommand{\headrulewidth}{0pt} % Remove a linha do cabeçalho
}
%%%%%% Redefinindo localização da paginação - FIM %%%%%%%%%%%%%%%%%%%%%%%%



%%%%% Removendo da contagem as a capa e a ficha catalográfica - INICIO  %%%%%%%%%%%%%%%%%%%%%%%%
\addtocounter{page}{-2} 
%%%%% Removendo da contagem as a capa e a ficha catalográfica - FIM  %%%%%%%%%%%%%%%%%%%%%%%%



%%%%% Configuração do espacamento padrão do documento - INICIO  %%%%%%%%%%%%%%%%%%%%%%%%
\usepackage{setspace}  % pacote usado para define o espacamento entre linhas do texto e de partes especificas
\onehalfspacing % definindo espaçamento padrao de 1,5 nas entrelinhas
%%%%% Configuração do espacamento padrão do documento - FIM  %%%%%%%%%%%%%%%%%%%%%%%%



%%%%% Configurações de links das referencias internas e externos - INICIO  %%%%%%%%%%%%%%%%%%%%%%%%
\usepackage{hyperref} % Cria os links das referencias internas e para links externos tb
\makeatletter
\hypersetup{
    pdftitle={\@title}
}
\makeatother
%%%%% Configurações de links das referencias internas e externos - FIM  %%%%%%%%%%%%%%%%%%%%%%%%