\usepackage[brazil]{babel}              % Traduções utilizados por vários pacotes
\usepackage[top=3cm,                    % definição padrao das margens das paginas
            bottom=2cm,
            left=3cm,
            right=2cm]{geometry}
\usepackage[T1]{fontenc}		        % Selecao de codigos de fonte.
\usepackage[utf8]{inputenc}		        % Codificacao do documento UTF-8 
\usepackage{PTSerif}                    % Pacote da fonte escolhida: Opções de fontes em: https://ctan.org/topic/font
\usepackage{graphicx}		            % Inclusão de gráficos, pdfs
\usepackage{lscape}                     % permite colocar um conteudo em formato paisagem (rotacionar)
\usepackage{color}                      % Permite utilizar cores pelos nomes
\usepackage{microtype}     			    % para melhorias de justificação
\usepackage[indentfirst=false]{quoting} % Usado para citação
\usepackage{url}                        % Usado para formatar URLs no texto
\usepackage{enumitem}                   % Permite alterar o formato das listas enumeradas 
\usepackage{booktabs}                   % Usado para personalizar elementos em tabelas, como \toprule e \midrule
\usepackage{arydshln}                   % Usado para personalizar elementos em tabelas, como \cdashline
\usepackage{amsmath}                    % Usado nas referencias de equações, através do comando \eqref
\usepackage{float}                      % Necessario para posicionar as figuras e tabelas de forma flutuante
\usepackage{enumitem}                   % Utilizada para formatar uma lista enumerada 
\usepackage{svg}                        % Pacote que permite utilização de imagens SVGs (vetoriais) no documento 

\usepackage{adjustbox}



%%%%% Redefinindo localização da paginação - INICIO %%%%%%%%%%%%%%%%%%%%%%%%
\usepackage{fancyhdr} % Pacote utilizado para alterar a posicao da paginação
\pagestyle{fancy}
\fancyhf{} % Limpa todos os cabeçalhos e rodapés padrão
\fancyhead[R]{\thepage} % Coloca o número da página no canto superior direito
\renewcommand{\headrulewidth}{0pt} % Remove a linha do cabeçalho (opcional)
\fancypagestyle{plain}{
    \fancyhf{} % Limpa cabeçalhos e rodapés
    \fancyhead[R]{\thepage} % Número da página no canto superior direito
    \renewcommand{\headrulewidth}{0pt} % Remove a linha do cabeçalho
}
%%%%%% Redefinindo localização da paginação - FIM %%%%%%%%%%%%%%%%%%%%%%%%



%%%%% Removendo da contagem as a capa e a ficha catalográfica - INICIO  %%%%%%%%%%%%%%%%%%%%%%%%
\addtocounter{page}{-2} 
%%%%% Removendo da contagem as a capa e a ficha catalográfica - FIM  %%%%%%%%%%%%%%%%%%%%%%%%



%%%%% Configuração do espacamento padrão do documento - INICIO  %%%%%%%%%%%%%%%%%%%%%%%%
\usepackage{setspace}  % pacote usado para define o espacamento entre linhas do texto e de partes especificas
\onehalfspacing % definindo espaçamento padrao de 1,5 nas entrelinhas
%%%%% Configuração do espacamento padrão do documento - FIM  %%%%%%%%%%%%%%%%%%%%%%%%



%%%%% Configurações de links das referencias internas e externos - INICIO  %%%%%%%%%%%%%%%%%%%%%%%%
\usepackage{hyperref} % Cria os links das referencias internas e para links externos tb
\makeatletter
\hypersetup{
    pdftitle={\@title}
}
\makeatother
%%%%% Configurações de links das referencias internas e externos - FIM  %%%%%%%%%%%%%%%%%%%%%%%%



%%%%% Configurações do texto de Fonte das imagens - INICIO  %%%%%%%%%%%%%%%%%%%%%%%%
\usepackage{copyrightbox} % Necessario para adicionar origem das imagem e das tabelas
\makeatletter
\renewcommand{\CRB@setcopyrightfont}{%
\color{black}
}
\makeatother
%%%%% Configurações do texto de Fonte das imagens - FIM  %%%%%%%%%%%%%%%%%%%%%%%%



%%%%% Evita reiniciar numeracoes das equações nos capitulos - INICIO  %%%%%%%%%%%%%%%%%%%%%%%%
\usepackage{chngcntr} % Permite modificar a numeração dos contadores: Equacoes, Tabelas, e Figuras
\counterwithout{equation}{chapter} % Remove a reinicialização da numeração a cada capítulo
% A configuração de tabela e figura estão em  seus respectivos arquivos de listagem 
% juntamente com as suas demais configurações 
%%%%% Evita reiniciar numeracoes das equações nos capitulos - FIM  %%%%%%%%%%%%%%%%%%%%%%%%

%%%%% Formatação da lista de figura e de tabelas - INICIO  %%%%%%%%%%%%%%%%%%%%%%%%
\usepackage{titletoc} % Pacote para formatar sumário, lista de figuras e tabelas

%a configuração só funciona se for antes do documentclass

%Lista de Tabelas
\titlecontents{figure}[0pt]
{}
{\figurename\ \thecontentslabel\ - }
{}
{\titlerule*[0.4pc]{.}\contentspage}
[\addvspace{5pt}]

%Lista de Tabelas
\titlecontents{table}[0pt]
{}
{\tablename\ \thecontentslabel\ - }
{ }
{\titlerule*[0.4pc]{.}\contentspage}
[\addvspace{5pt}]
%%%%% Formatação da lista de figura e de tabelas - FIM  %%%%%%%%%%%%%%%%%%%%%%%%



%%%%% Formatação dos titulos dos capitulos, secao, subsecao, etc - INICIO  %%%%%%%%%%%%%%%%%%%%%%%%
\usepackage{titlesec}

\titlespacing{\chapter}
  {0pt}   % Espaço à esquerda
  {-17pt}   % Espaço acima do título do capítulo (é aqui que se reduz)
  {15pt}  % Espaço abaixo do título do capítulo

\titleformat{\chapter}[hang] % shape
{\normalsize\bfseries\uppercase} % format
{\thechapter}
{10pt}
{}

\titleformat{\section}[hang] % shape
{\normalsize\mdseries\uppercase} % format
{\thesection}
{10pt}
{}

\titleformat{\subsection}[hang] % shape
{\normalsize\bfseries} % format
{\thesubsection}
{10pt}
{}

\titleformat{\subsubsection}[hang] % shape
{\normalsize\bfseries\itshape} % format
{\thesubsubsection}
{10pt}
{}

\titleformat{\paragraph}[hang] % shape
{\normalsize\itshape} % format
{\theparagraph}
{10pt}
{}

\titleformat{\subparagraph}[hang] % shape
{\normalsize\mdseries} % format
{\thesubparagraph}
{}
{}
%%%%% Formatação dos titulos dos capitulos, secao, subsecao, etc - FIM  %%%%%%%%%%%%%%%%%%%%%%%%



%%%%% Formatação das citações padrão ABNT - INICIO  
\usepackage{abntex2cite} % citações no padrão ABNT
%%%%% Formatação das citações padrão ABNT - FIM